\thispagestyle{empty}

\vbox to\textheight{

\hbox to\textwidth{%
\begin{tabular}[t]{@{}l@{ }l@{}}
%УДК&512.763+512.732\\% для математики находится по (неполной?) таблице УДК в~201
%ББК&22.147\\% находится по таблице ББК в~201
%&А61% находится по таблице авторских знаков в~201
\end{tabular}
\hss}

\vspace{3\baselineskip}

%\hyphenation{МЦНМО}


{\settowidth{\leftskip}{А61\hskip1em}
\ifdim\overfullrule>0pt
\vbox to 0pt{\noindent\hskip-\leftskip\hskip-.5em \vrule height 6cm\vss}\fi
\noindent\textbf{Е. К. Кузьмичев}

%\noindent\hbox to 0pt{\hskip-\leftskip А61\hss}\hskip\parindent

Е. К. Кузьмичев

\vspace{2pt}
%{\small
%ISBN 978-5-94057-572-6}

\vskip.5\baselineskip

{\footnotesize
Настала необходимость издать книгу поэзии крестьянского поэта Егора Кузьмича Кузьмичева (1867--1933), т.к. последнее крупное издание было в 1917 году. Пройдя сквозь век стихи приобрели ценность как документ эпохи. 
Его сугубо крестьянская лирика обращена к земле и считается продолжением крестьянской линии поэзии Сурикова, Никитина, Кольцова.
В поэзии крестьянского поэта сильна связь с народной песней. Стиль, мелодия, ритм – все звучит русским звуком.
Сохранилась книга 1917 года с рукописными авторскими правками. Составители издают обновленный текст стихов по авторским правкам.
\par
\vskip\baselineskip}

%\hbox to\textwidth{\hfil ББК 22.147}

}

\vfill

\hbox to\textwidth{%
\begin{tabular}[b]{@{}l}
%\textbf{ISBN 978-5-94057-572-6}
\end{tabular}
\hfill
\begin{tabular}[b]{@{}l@{ }l@{}}
%\copyright
%Е. К. Кузьмичев
%\copyright
%&МЦНМО, 200?
\end{tabular}}}
\clearpage
